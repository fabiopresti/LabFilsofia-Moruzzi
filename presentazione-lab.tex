\documentclass[10pt,a4paper]{article}
\usepackage[utf8]{inputenc}
\usepackage[T1]{fontenc}
\usepackage[italian]{babel}
\usepackage{amsmath}
\usepackage{amsfonts}
\usepackage{amssymb}
\usepackage{makeidx}
\author{Fabio Prestipino}
\title{
	Il rapporto tra stile e contenuto in Wittgenstein}

\begin{document}
	\maketitle
\begin{abstract}
	Leggere Wittgenstein è indubbiamente un’impresa ardua e parte della difficoltà è probabilmente legata allo stile, da alcuni definito “oracolare” (fra cui L. Goldstein, 1999). Lo sviluppo degli argomenti è sconnesso, suddiviso in paragrafi numerati di lunghezza variabile, spesso gli argomenti non sono esposti linearmente ma sparsi lungo l’intera opera riproponendo stessi punti sotto diverse luci. Molto si è detto sulle ragioni che hanno spinto Wittgenstein a questa scelta e anche Saul Kripke, in relazione alle Ricerche, offre una spiegazione coerente con la sua interpretazione dell'opera (\textit{Wittgenstein su regole e linguaggio privato}). A partire dall’esposizione della tesi di Kripke, si approfondisce questo tema basandosi su quanto lo stesso Wittgenstein asserisce nelle Ricerche (le informazioni sono principalmente tratte da \textit{Introduzione a Wittgenstein, L. Perissinotto, Il Mulino, 2018 pp. 23-28}).
\end{abstract}
\section{La versione di Kripke}
Per Kripke Wittgenstein formula il problema scettico del significato e ne da una soluzione scettica. In altre parole Wittgenstein vuole dare ragione allo scettico nel sostenere che non esistono 
\begin{quote}
	Super fatti  che i filosofi associano in maniera fuorviante a tali espressioni correnti [“il fatto che Jones con il tale simbolo intendesse l’addizione”], e non negare che tali espressioni siano appropriate.
\end{quote}
Risulta evidente che seguire una tale linea argomentativa rasenta la contraddittorietà: se non esistono dei fatti che garantiscano che, ad esempio, l’uso passato o la mia comprensione della regola d’addizione determini univocamente il modo presente e futuro di performare un’addizione, come sarebbe possibile non rifiutare le credenze comuni? Per Kripke lo stile ostico delle Ricerche è proprio dovuto al fatto che, se si fossero esposte le tesi in modo sistematico, sarebbe stato difficile non cadere nella negazione scettica delle affermazioni ordinarie
\begin{quote}
	Quando il nostro avversario insiste che una comune forma di espressione è perfettamente appropriata, possiamo ribattere che \textit{se queste espressioni sono intese correttamente}, siamo d’accordo anche noi. Il pericolo si presenta quando cerchiamo di dare una formulazione precisa di ciò che effettivamente stiamo proprio negando (Wittgenstein su regole e linguaggio privato).
\end{quote} 
Se infatti dovessimo proporre una teoria filosofica sistematica staremmo sostenendo un asserto che sarebbe perfettamente accettabile, \textit{se inteso correttamente}. Ed ecco che si cade nuovamente nei meandri dei problemi relativi al significato! Vediamo una volta di più la natura eterea e sfuggente di questo paradosso.\\
Inserita in un quadro più ampio, la tesi kripkeana appare ancor più convincente. Wittgenstein intendeva il suo lavoro come un modo nuovo di fare filosofia
\begin{quote}
	Sebbene quello che facesse fosse certamente diverso da quello che avevano fatto, per esempio, Platone o Berkeley, tuttavia si poteva avere l’impressione che il suo tipo di ricerca “prendesse il posto di” ciò che avevano fatto Platone o Berkeley (Libro Marrone)
\end{quote}
Questa nuova attività rompeva non tanto con i contenuti quanto più nel metodo e modo di pensare. La prefazione alle Ricerche offre un punto di vista interessante: inizialmente Wittgenstein vede come un fallimento l’adozione di questo stile
\begin{quote}
	In principio era mia intenzione raccogliere tutte queste cose in un libro, la cui forma immaginavo di volta in volta diversa. Essenziale mi sembrava, in ogni caso, che i pensieri dovessero procedere da un soggetto all’altro secondo una successione naturale e continua. (Ricerche Filosofiche, pref.)
\end{quote}
Tuttavia, dopo “diversi infelici tentativi” si rende conto della necessità di questo esito:
\begin{quote}
	Non appena tentavo di costringere i miei pensieri in una direzione facendo violenza alla loro naturale inclinazione, subito questi si deformavano. E ciò dipendeva senza dubbio dalla natura stessa della ricerca […] (Ricerche Filosofiche, pref.)
\end{quote}
Leggendo queste righe con l’interpretazione di Kripke in mente, ricordando che poche frasi prima Wittgenstein stesso individua nel significato il primo dei temi di cui le Ricerche si interessano, sembra naturale pensare che "la natura stessa della ricerca" sia proprio il cosiddetto paradosso di Kripkenstein. Non è dunque possibile parlare sistematicamente del paradosso in virtù del paradosso stesso, lo stile è frutto dell’argomento trattato e si piega sotto la sua potenza.
\section{Il paragone con Berkeley}
Il paragone con Berkeley che propone Kripke è calzante. Il filosofo irlandese è qui citato in relazione alla sua tesi secondo cui nulla esiste al di fuori della mente: non esiste la materia, ma solo Dio e gli spiriti umani. Nel sostenere questa tesi in forma sistematica, com'era normale nel periodo storico in cui si colloca, Berkeley cade nell'analoga negazione scettica del senso comune che Wittgenstein vuole evitare. Per stemperare questo esito Berkeley non opera un cambiamento di stile ma sposta il discorso da un piano metafisico ad uno linguistico, giustificando il senso comune con una "\textit{interpretazione} metafisica erronea del modo comune di parlare": 
\begin{quote}
	Le idee che ci facciamo delle cose sono tutto ciò che possiamo dire della materia. Perciò per "materia" si deve intendere una sostanza inerte e priva di alcun senso, della quale però si pensa che abbia estensione, forma e movimento. È quindi chiaro che la nozione stessa di ciò che viene chiamato "materia" o "sostanza corporea" è contraddittoria. Non è quindi il caso di spendere altro tempo per dimostrarne l'assurdità.
\end{quote}
Questa soluzione non è acettabile agli occhi di Kripke, che osserva come una tale soluzione sia frequente in filosofia e non faccia altro che forzare un'interpretazione innaturale del linguaggio. Kripke osserva infatti come Berkeley ad un certo punto non possa far a meno di riconoscere che le sue tesi appaiano in contrapposizione all'uso comune delle parole. Si osservi che la questione è sempre relativa all'interpretazione: Berkeley sostiene che ,\textit{se intesa correttamente}, la parola "materia" è comunemente usata in modo corretto, ma quale fatto impone che bisogna seguire questa interpretazione? Il paradosso scettico torna ad insediare il discorso filosofico.
\section{Lo stile nel Tractatus}
Il problema dello stile non si limita alle Ricerche ma è presente anche nel Tractatus, che lo stesso Frege trova “difficile da capire” (fatto emblematico della distanza delle categorie Fregeane da quelle di Wittgenstein). Se l’organizzazione in proposizioni può apparire in prima battuta un tentativo di rigorosità, di irregimentare il linguggio, Wittgenstein stesso ci contraddice: Il lavoro è rigidamente filosofico e insieme letterario (Lettera a Ludwig von Ficker, 1919). Ed in questo contesto l’aggettivo “letterario” potrebbe alludere alla non univocità delle interpretazioni, ad una negazione di ogni pretesa di sistematicità e rigorosità.
\section{Wittgenstein e Nietzsche?}
In conclusione, sperando di non andar troppo fuori tema, mi sembra interessante stimolare il discorso proponendo un’analogia nel rapporto fra contenuto e stile in Nietzsche e Wittgenstein: entrambi i filosofi sono percepiti come un unicum nella storia della filosofia, in rottura con le tradizioni, entrambi intendono la loro attività come un nuovo modo di far filosofia e rifiutano ogni sistematicità, ritenuta impossibile e fuorviante. Nonostante i motivi che li portano a queste conclusioni siano profondamente diversi (o forse è possibile trovare qualche analogia più profonda?) il corollario che fanno seguire è lo stesso: una scrittura aforismatica, sconnessa ed oracolare a tratti. È interessante rileggere il rifiuto di una distinzione netta tra letteratura e filosofia  da parte di Wittgenstein alla luce del ruolo che l’arte svolge nella pensiero nietzschiano: per il primo il rifiuto mi sembra sia dovuto all’impossibilità di applicare una logica rigorosa al linguaggio (unico modo per evitare insidiosi paradossi) mentre per il secondo questa stessa generale pretesa di rigore razionale è l’errore più grande di tutta la filosofia occidentale. Questo errore può portare, per Nietzsche, a conseguenze catastrofiche per tutta la società e l’unico riparo possibile è offerto dall’abbandono nell’arte.
\end{document}